% !TEX encoding = UTF-8 Unicode
% !TEX TS-program = pdflatex
% !TeX spellcheck = it-IT
% Preambolo
\documentclass[a4paper,12pt]{article}
\pagestyle{plain}
\usepackage[T1]{fontenc}
\usepackage[utf8]{inputenc}
\usepackage{geometry}
\usepackage{amssymb}
\usepackage{amsmath} % per i simboli matematici
\geometry{a4paper,top=2.5cm,bottom=2.5cm,left=2cm,right=2cm,heightrounded,bindingoffset=5mm}
\usepackage{microtype} % per migliorare il riempimento delle righe
\usepackage{setspace} % per l'interlinea
\onehalfspacing
\usepackage[english, italian]{babel}
\newcommand\resw[1]{\mathtt{#1}}
\newcommand\mi[1]{\mathit{#1}}
\title{Progetto Compilatori \\ \normalsize{A.A. 2020/2021}}
\author{Gaetano Antonucci \and Alessio Romano}
\date{\today}
\begin{document}
    \maketitle
    \tableofcontents
    \newpage
    
    \section{Regole di Type Checking implementate}
    \subsection{Tipi Primitivi}
    \[
        \Gamma \vdash null \colon null \qquad
        \Gamma \vdash true \colon boolean \qquad
        \Gamma \vdash false \colon boolean \qquad
    \]
    \[
        \Gamma \vdash \mi{int} \colon int \qquad
        \Gamma \vdash \mi{float} \colon float \qquad
        \Gamma \vdash \mi{string} \colon string \qquad
        \Gamma \vdash \mi{bool}    \colon boolean \qquad
    \]
    \subsection{Dichiarazioni di Variabili}
    \[
        \frac{(x \colon \tau) \in \Gamma}{\Gamma \vdash x \colon \tau}
    \]
    \subsection{Operazioni Unarie}
    \[
        \frac {\Gamma \vdash \mi{e} \colon \tau_1 \quad optype1(op,\tau_1) = \tau} %
              {\Gamma \vdash (op \, \mi{e}) \colon \tau}
    \]
    \subsection{Operazioni Binarie}
    \[
        \frac {\Gamma \vdash \mi{e_1} \colon \tau_1 \quad \Gamma \vdash \mi{e_2} \colon \tau_2  \quad optype2(op,\tau_1, \tau_2) = \tau} %
        {\Gamma \vdash (\mi{e_1}\, op \, \mi{e_2}) \colon \tau}
    \]
    \subsection{Chiamata a Procedura}
    \[
        \frac {\Gamma \vdash \mi{f} \colon \tau_i^{\,i \, \in \,1 \, \dots \, n} \to \tau_j^{\,j \, \in \, 1 \, \dots \, m} \quad \Gamma \vdash \mi{e_i} \colon %
        \tau_i^{\, i \, \in \, 1 \, \dots \, n}} %
        {\Gamma \vdash \mi{f}(\mi{e}_i^{\, i \, \in \, 1 \, \dots \, n})\colon \tau_j^{\, j \, \in \, 1 \, \dots \, m}}
    \]
    
    \subsection{Statement}
    	\subsubsection{if-then}
	\[
		\frac{\Gamma \vdash \mi{e} \colon boolean \quad \Gamma \vdash \mi{stmt}}%
		{\Gamma \vdash \resw{if} \, \mi{e} \, \resw{then} \, \mi{stmt} \, \resw{fi}}
	\]
	\subsubsection{if-then-else}
	\[
		\frac{\Gamma \vdash \mi{e} \colon boolean \quad \Gamma \vdash \mi{stmt}_{1} \quad \Gamma \vdash \mi{stmt}_{2}}%
		{\Gamma \vdash \resw{if} \, \mi{e} \, \resw{then} \, \mi{stmt}_{1} \, \resw{else} \, \mi{stmt}_{2} \, \resw{fi}}
	\]
	\subsubsection{if-then-elif-else}
	\[
		\frac{\Gamma \vdash \mi{e}_{j}^{\, j \, \in \, 1 \, \dots \, m} \colon boolean \quad %
		\Gamma \vdash \mi{stmt}_{i}^{i \, \in \, 1 \, \dots \, 3}}%
		{%denominatore
		\Gamma \vdash  \resw{if} \, \mi{e}_{1} \, \resw{then} \, \mi{stmt}_{1}  \,
		 (\resw{elif} \, \mi{e}_{j}^{\, j \, \in \, 2 \, \dots \, m} \, \resw{then} \, \mi{stmt}_{2}\,)_{t}^{t \, \in \, 1 \, \dots \, k} \,
		  \resw{else} \, \mi{stmt}_{3} \, \resw{fi}
		}%
	\]
	\subsubsection{while}
	\[
		\frac{\Gamma \vdash \mi{e} \colon boolean \quad \Gamma \vdash \mi{stmt}}%
		{\Gamma \vdash \resw{while} \, \mi{e} \, \resw{do}\, \mi{stmt} \, \resw{od}}
	\]
	\subsubsection{while-return}
	\[
		\frac{\Gamma \vdash \mi{e} \colon boolean \quad \Gamma \vdash \mi{stmt}_{1} \quad \Gamma \vdash \mi{stmt}_{2}}%
		{\Gamma \vdash \resw{while} \, \mi{stmt}_{1}  \resw{-\!\!>}\, \mi{e} \, \resw{do}\, \mi{stmt}_{2} \, \resw{od}}
	\]
	\subsubsection{readln}
	\[
		\frac{(x_{i}^{\, i \, \in \, 1 \, \dots \, n} \colon \tau_{i}^{\, i \, \in \, 1 \, \dots \, n}) \, \in \, \Gamma}%
		{\Gamma \vdash \resw{readln(} \mi{x}_{i}^{\, i \, \in \, 1 \, \dots \, n} \resw{)}}
	\]

\end{document}